% !TEX encoding = UTF-8
% !TEX TS-program = pdflatex
% !TEX root = ../tesi.tex
% !TEX spellcheck = it-IT

%**************************************************************
\chapter{Realizzazione}
\label{cap:realizzazione}
%**************************************************************

Questo capitolo contiene la descrizione delle attività svolte e dei problemi incontrati durate lo sviluppo dell'applicazione, che è iniziato a partire dal prototipo realizzato durante lo studio di React Native.

L'attività di codifica è stata organizzata in modo da riuscire a produrre delle versioni intermedie dell'applicazione, da utilizzare per effettuare delle demo all'interno dell'azienda e per valutare la necessità di alcune modifiche.

\section{Dal prototipo alla gallery}

Prima di iniziare lo sviluppo di nuove funzionalità sul prototipo è stato prima necessario adattarlo all'architettura progettata modificando alcuni componenti.

La versione del prototipo che è stata utilizzata implementava già il pattern Flux mediante il modulo npm \texttt{flux} ed era organizzata con due \textit{stores}, uno che gestiva le immagini e uno che si occupava di gestire le fonti disponibili per le immagini.

Come prima cosa sono stati modificati gli \textit{stores} e le relative \textit{actions} secondo quando progettato, trasformando cioè lo \textit{store} delle immagini in quello degli assets e lo \textit{store} delle fonti in quello dei nodi.

Inoltre, l'oggetto del prototipo che si occupava del recuperare i dati è stato modificato in modo che interroghi le API di WARDA e che gestisca l'autenticazione.

L'implementazione dell'autenticazione è stata semplice in quanto è bastato aggiungere una variabile boolena che specifica se l'autenticazione è già stata effettuata. Nel caso questa non sia già stata effettuata, viene prima richiesta l'autenticazione e poi viene effettuata la chiamata alle API vera e propria.

Non è stato necessario gestire cookies o quant'altro dal momento che l'oggetto \texttt{fetch} offerto da React Native per effettuare chiamate HTTP funziona come proxy del componente nativo di iOS che si occupa di gestire il traffico di rete e questo componente gestisce in modo automatico i cookies.

Per quanto riguarda l'interfaccia grafica è stato necessario modificare la griglia in modo che visualizzi oggetti di tipo \texttt{Asset} e che implementi lo scroll infinito.

La griglia degli assets è stata implementata utilizzando il componente \texttt{ListView} di React Native, il quale rende disponibile l'evento \texttt{onEndReached}, che viene sollevato ogni volta che l'utente raggiunge la fine della lista e che permette implementare lo scroll infinito. 

Tuttavia questa implementazione aveva due problemi:
\begin{itemize}
\item il caricamento dei nuovi dati iniziava quanto l'utente arrivava alla fine delle lista, quando poteva essere anticipato, in modo da migliorare ulteriormente l'esperienza d'uso;
\item in alcuni casi l'evento veniva sollevato troppe volte e questo causava il caricamento ripetuto dello stesso blocco di dati.
\end{itemize}

Questi problemi sono stati risolti implementano lo scroll infinito utilizzando l'evento \texttt{onScroll} della \texttt{ListView}, evento che viene sollevato ogni volta che l'utente esegue uno scroll.

Alla funzione che gestisce tale evento viene passato come parametro un oggetto con tutte le informazioni relative all'evento che si è verificato, in questo modo è stato possibile richiedere il caricamento di ulteriori dati ad una distanza dalla fine personalizzata ed effettuare maggiori controlli in modo da evitare che la stessa porzione di dati venga caricata più volte.

\begin{lstlisting}[language=JavaScript, caption=Funzione che gestisce l'evento onScroll della griglia che visualizza gli assets]
//AssetsGrid.js
function onScroll(args) {
  var scrollY = args.nativeEvent.contentOffset.y;
  var height = args.nativeEvent.contentSize.height;
  var visibleHeight = args.nativeEvent.layoutMeasurement.height;

  if (scrollY > height - 2 * visibleHeight){
    //Se mancano meno di due schermate alla fine della lista, richiedo il caricamento di ulteriori dati
    //Per evitare che il caricamento venga effettuato troppe volte controllo che i dati siano cambiati rispetto all'ultima volta che ho effettuato il caricamento.
    if (_dataChanged){ //Variabile che viene settata a true quando il componente riceve dei dati nuovi
      _dataChanged = false;
      var downloadStarted = this.props.onLoadMoreData();
      var temp = this.props.assets.concat([{loading:true}]); //Visualizza un'indicatore di caricamento
      this.setState({
        downloading: downloadStarted,
        dataSource: dataSource.cloneWithRows(temp)
      });
    }
  }
}
\end{lstlisting}

Una volta completata la visualizzazione a griglia è stato implementato il popover con l'anteprima dei dettagli, il quale è stato realizzato utilizzando il componente \texttt{react-native-popover}\footnote{\url{https://github.com/jeanregisser/react-native-popover}} presente in npm. 
Questo componente permette di visualizzare un popover contenente altri componenti di React Native ed è stato utilizzato per contenere la lista dei dettagli di un asset.

Al termine di questa prima fase l'applicazione è in grado di visualizzare il contenuto di un nodo e di visualizzare la lista dei nodi disponibili, anche se non è ancora possibile navigare tra i vari nodi.

\section{Sistema di navigazione}

Successivamente è stato sviluppato il sistema di navigazione, il quale fornisce la possibilità all'utente di scendere o risalire lungo la gerarchia dei nodi.

La maggior parte delle modifiche effettuate in questa fase riguardano \texttt{NodesStores} in quanto è stato necessario implementare un sistema che tenga traccia del percorso effettuato dall'utente lungo la gerarchia della gallery.

Questo sistema è inoltre vincolato dal funzionamento delle API di WARDA, le quali, dato l'id di un nodo, permettono solamente di ottenere le informazioni riguardanti i nodi figli o gli assets in esso contenuti.

\`E stato quindi necessario implementare un sistema che, quando viene richiesto il caricamento dei nodi a partire da un determinato URL, sia in grado di recuperare un oggetto \texttt{Node} contenente le informazioni relative al nodo padre dei nodi di cui è richiesto il caricamento.
Inoltre, questo sistema deve essere in grado di tenere traccia del percorso che l'utente ha effettuato durante la navigazione della gallery.

Per far funzionare il tutto è stato necessario modificare la funzione che carica i nodi all'interno dello \textit{store} in modo che, prima di caricare i nuovi nodi, cerchi tra gli oggetti \texttt{Node} presenti all'interno dello \textit{store} l'oggetto rappresentante il nodo padre dei nuovi nodi utilizzando l'URL dal quale sono stati scaricati i nuovi dati.
L'oggetto viene poi inserito in coda ad un'array contenente tutti i nodi appartenenti al ramo della gallery che l'utente sta visualizzando.

Nel caso che l'utente stia tornando all nodo precedente, anziché effettuare l'inserimento dell'oggetto nodo, viene rimosso l'ultimo elemento dell'array. 

\begin{lstlisting}[language=JavaScript, caption=NodesStore - Caricamento dei nodi]
//NodesStore.js

//_nodes --> Array con i nodi correntemente visualizzati
//_nodeHierarchy --> Array con i nodi visualizzati dall'utente, è inizializzato come array vuoto
function loadNodes(parentUrl, nodes) {

  if (_nodesHierarchy.length === 0){
    //E' il primo caricamento dei dati, la gerarchia dei nodi inizia da un nodo "finto" che funziona da radice
    //in quanto le API di WARDA non forniscono informazioni riguardanti il nodo radice di una gallery
    _nodesHierarchy.push({isRoot:true,text:'', urlContentDataStore:'', urlChildrenStore:''});
  } else {
    //Ultimo nodo del ramo della gerarchia dei nodi che l'utente ha visualizzato
    var lastNode = _nodesHierarchy[_nodesHierarchy.length -1];

    if (lastNode.urlChildrenStore.length < parentUrl.length){
     //Se l'url dell'ultimo nodo è più corto (più alto in gerarchia) dell'url del nodo corrente, vuol dire che l'utente sta scendendo lungo il ramo
      //Il nodo corrente (quello di cui sto caricando i figli) è il nodo che ha urlContentDataStore == parentUrl e che si trova correntemente in memoria
      var currentNode = _nodes.filter((item) => item.urlChildrenStore === parentUrl)[0];
     //Aggiungo il nodo corrente in coda
      _nodesHierarchy.push(currentNode);
    } else {
      //L'url dell'ultimo nodo è più lungo dell'url del nodo corrente, vuol dire che sto risalendo lungo il ramo
      //Faccio un pop per togliere l'ultimo elemento dell'array
      _nodesHierarchy.pop();
    }
  }
  //Aggiornamento dei nodi
  _nodes = nodes;
}
\end{lstlisting}

Questa versione dell'applicazione rispecchia gli obiettivi minimi previsti dal piano di stage ed è risultata particolarmente fluida, considerando anche che non è stata effettuata alcun tipo di ottimizzazione per quanto riguarda la gestione della memoria.

\section{Sistema dei filtri}

Una volta completati gli obiettivi minimi è stato implementato il sistema dei filtri per il contenuto di un nodo.

La struttura di questo sistema risulta particolarmente complessa, in quanto i valori che può assumere un filtro dipendono sia dal nodo sul quale viene applicato il filtro, sia dai filtri che sono già stati applicati.

Fortunatamente le API di WARDA forniscono un livello di astrazione tale da rendere l'implementazione lato client semplice.

Infatti, per inserire il sistema di filtri all'interno dell'applicazione è bastato implementare \texttt{FiltersStore} per tenere traccia dei filtri applicati e modificare alcuni metodi di \texttt{AssetsActions} e di \texttt{WardaFetcher} in modo che prendessero un ulteriore parametro con i filtri da applicare.

\begin{lstlisting}[language=JavaScript, caption=WardaFetcher - Caricamento degli assets considerando i filtri]
//WardaFetcher.js
function fetchContents(contentUrl, start, filters){
  //Funzione che scarica gli assets contenuti in un nodo

  var filtersString = '';
  //Gestione dei casi limite
  if (!contentUrl){ contentUrl = ''; }
  if (!start){ start = 0; }

  if (filters && filters.length > 0){
    //Conversione dell'array con i filtri da applicare in stringa
    filtersString = JSON.stringify(filters);
    //Codifica della stringa
    filtersString = '&filter='+encodeURIComponent(filtersString);
  }
  //Template string, una nuova tipologia di stringhe presente nello standard ES6 di JavaScript, le variabili presenti all'interno del blocco ${ } vengono sostituite con il loro valore
  var url=`${WARDA_URL+contentUrl}?start=${start}&limit=${Config.PAGE_SIZE}${filtersString}`;

  //La richiesta effettiva dei dati viene effettuata in modo autenticato
  return authenticatedFetch(url)
    .then((response) => response.json());
}
\end{lstlisting}

Per quanto riguarda l'implementazione grafica è stato riutilizzato il componente \texttt{react-native-popover}.

\section{Visualizzazione di dettaglio}

Durante questa fase è stato implementato il componente \texttt{AssetDetailPage} che viene visualizzato quanto l'utente effettua un tap sull'immagine di asset nella visualizzazione a griglia. 

Una caratteristica di questa pagina è che quando l'utente esegue uno swipe sull'immagine dell'asset, devono essere visualizzati i dettagli dell'asset precedente o successivo in base alla direzione dello swipe, ottenendo così una visualizzazione della gallery a carosello.

L'introduzione di questa tipologia di visualizzazione ha sollevato alcune problematiche:
\begin{itemize}
\item l'ordine di visualizzazione degli assets nel carosello deve corrispondere all'ordine degli assets presenti nella griglia di \texttt{GalleryPage};
\item nel client non sempre sono presenti tutti gli assets contenuti nel nodo selezionato ed è quindi necessario che anche dalla visualizzazione a carosello sia possibile effettuare il download di ulteriori assets in un modo analogo allo scroll infinito.
\end{itemize}

In questo caso si sono visti alcuni dei vantaggi dell'architettura Flux, in quanto è bastato alimentare \texttt{AssetDetailPage} con i dati presenti in \texttt{AssetsStore} e aggiungere due metodi a quest'ultimo per permettere di recuperare l'asset precedente o successivo a partire da un determinato asset, risolvendo così il problema relativo all'ordine di visualizzazione.

Così facendo è stato possibile utilizzare anche la stessa \textit{action} che utilizza \texttt{GalleryPage} per caricare ulteriori assets.
In questo caso l'\textit{action} viene eseguita quando è richiesta la visualizzazione dell'asset successivo e questo non è presente nel client.

Per controllare l'esistenza dell'asset successivo viene confrontato il numero di assets presenti sul client con il numero di assets presenti nel nodo corrente e solo nel caso che ci sia un'asset successivo viene richiesto il download, diminuendo così il traffico di rete.

Questa implementazione permette anche di limitare il numero di download, infatti, quando viene richiesto il caricamento di ulteriori dati dalla visualizzazione a carosello, questi diventano disponibili anche per la griglia di \texttt{GalleryPage}.

Una volta ultimato il carosello, è stato implemento il componente \texttt{NetworkImage} in modo che l'utente riceva un feedback riguardo il caricamento dell'immagine.
Questo si è reso necessario dal momento che l'immagine visualizzata in questa pagina è ad alta risoluzione e il più delle volte il download richiede una quantità di tempo significativa.

L'implementazione di questo componente ha sofferto per un po' di tempo di alcuni problemi che derivavano da un bug del componente \texttt{Image} di React Native.
La versione \texttt{0.9.0} del framework ha poi risolto questi bug, rendendo così possibile l'implementazione finale di \texttt{NetworkImage}.


\section{Animazioni}

Una volta soddisfatti i requisiti obbligatori e desiderabili si è passati all'introduzione di alcune animazioni sfruttando le API offerte da React Native che hanno permesso di ottenere animazioni fluide con delle prestazioni paragonabili a quelle delle animazioni native.

Le prime animazioni introdotte riguardano la comparsa e la scomparsa della lista dei nodi nella visualizzazione della gallery e della lista dei dettagli di un asset nella visualizzazione di dettaglio.

Queste animazioni sono state ottenute utilizzando le API \texttt{LayoutAnimation} che permettono di renderizzare l'interfaccia grafica in modo che le modifiche risultino animate.
\begin{lstlisting}[language=JavaScript, caption=AssetDetailPage - Animazione della comparsa/scomparsa lista dei dettagli]
//AssetDetailPage.js
function onDetailButtonPress() {
  LayoutAnimation.easeInEaseOut(); //Imposta l'animazione
  this.setState({detailsVisible: !this.state.detailsVisible}); //Modifica lo stato del componente causandone il re-rendering
}
\end{lstlisting}
Come si può notare dall'esempio sopra riportato l'implementazione di queste animazioni risulta estremamente semplice.

Una volta inserite tutte le animazioni necessarie alla modifica del layout è stato animato lo swipe della visualizzazione a carosello di \texttt{AssetDetailPage}.

Questa animazione ha richiesto l'utilizzo delle API \texttt{Animated} che permettono di definire animazioni più complesse.

Per funzionare, queste API utilizzano dei particolari componenti grafici, il cui stile dipende da determinati valori che possono essere modificati in vario modo. La modifica di questi valori comporta quindi la modifica dello stile del componente in modo animato.

Ad esempio, il seguente codice implementa l'animazione dell'immagine in modo che l'immagine segua lo swipe dell'utente.

\begin{lstlisting}[language=JavaScript, caption=AssetDetailPage - Spostamento dell'immagine allo swipe delll'utente]
//AssetDetailPage.js - Costruttore
...
this.state.panX = new Animated.Value(0); //Variabile che rappresenta lo spostamento dell'immagine
this.state.swipePanResponder = PanResponder.create({ //Oggetto che si occupa di rilevare le gesture dell'utente
  ...
  onPanResponderMove: Animated.event([null, {dx: this.state.panX}]), //All'evento onPanResponderMove, che viene sollevato quando l'utente esegue un pan (equivalente del drag'n'drop nei dispositivi touchscreen) viene collegata la variabile panX, in modo che il valore della variabile venga modificato e che la modifica venga effettuata in modo animato
  ...
});

//funzione di rendering dell'immagine
return (
  <Animated.View
    {...this.state.swipePanResponder.panHandlers} //Il gestore degli eventi viene collegato alla View
    style={..., {
      transform: [{translateX: this.state.panX},  ], //Il valore dello spostamento viene associato allo stile della View, in particolare alla traslazione sull'asse X
    }]}>
    	//Codice che visualizza l'immagine
  </Animated.View>
);
\end{lstlisting}

L'effetto prodotto dal quel codice è il seguente:

\vspace{1em}
\noindent
\begin{minipage}{\textwidth}
  \begin{minipage}[b]{0.49\textwidth}
    \centering
    \includegraphics[scale=0.25]{../immagini/swipe-start}
      \captionof{figure}{Immagine all'inzio dello swipe}
	\end{minipage}
	  \hfill
  \begin{minipage}[b]{0.49\textwidth}
    \centering
  \includegraphics[scale=0.25]{../immagini/swipe-follow}
      \captionof{figure}{Immagine durante lo swipe}
  \end{minipage}
\end{minipage}

\vspace{1em}
Tuttavia, il codice sopra riportato sposta l'immagine in base al pan effettuato dall'utente e non esegue nessuna animazione quando lo swipe viene completato e viene visualizzata un'altra immagine.

Nelle applicazioni native, una volta che l'utente ha completato uno swipe, l'immagine corrente viene mandata ``\textit{fuori dallo schermo}'' in modo animato per lasciare spazio ad una nuova immagine che entra dalla parte opposta.

Mentre, se l'utente non completa lo swipe, l'immagine viene fatta ritornare alla posizione iniziale in modo animato.

Per implementare ciò, è stato necessario modificare il gestore dello swipe, in modo che venga eseguita un'animazione che sposta l'immagine quando l'utente completa o annulla la gesture.

\`E stato quindi necessario modificare il codice precedentemente riportato aggiungendo le nuove animazioni:
\begin{lstlisting}[language=JavaScript, caption=AssetDetailPage - Animazione dello swipe]
//AssetDetailPage.js - Costruttore
this.state.swipePanResponder = PanResponder.create({
 ...
  onPanResponderRelease: (e, gestureState) => { //Funzione che viene invocata quando l'utente termina la gesture
    var swipeSize = 150; //Dimensione dello swipe in pixel
    
    //gestureState.dx rappresenta lo spostamento sull'asse X effettuato dall'utente
    if (Math.abs(gestureState.dx)>swipeSize){
      //L'utente ha completato lo swipe
      var toValue;
      
      if (gestureState.dx > 0) {
        toValue = 1000; //Valore sufficientemente grande in modo che l'immagine venga renderizzata fuori dallo schermo
      } else {
        toValue = -1000;
      }
     
      //Mando l'immagine offscreen in modo animato, la direzione dipende dalla direzione della gesture (segno di gestureState.dx)
      Animated.spring(this.state.panX, {
        toValue,
        velocity: gestureState.vx,
        tension: 10,
        friction: 3,
      }).start(); //Avvio dell'animazione
      
      this.state.panX.removeAllListeners();
      var id = this.state.panX.addListener(function({value}){ 
      	//Aggiungo un listener all'esecuzione dell'animazione in modo di riuscire a capire quando l'immagine è finita fuori dallo schermo
        if (Math.abs(value) > 400) { //L'immagine è fuori dallo schermo
          this.state.panX.removeListener(id);
          
          var loading = (value > 0)? this.showPreviousAsset() : this.showNextAsset();
          if (loading){
            //C'è una nuova immagine da visualizzare
            this.state.panX.setValue(-toValue); //Posiziona l'immagine dalla parte opposta dello schermo
            Animated.spring(this.state.panX, { //Animazione che fa "entrare" la nuova immagine dalla parte opposta dello schermo
              toValue:0,
              velocity: gestureState.vx,
              tension:1,
              friction:6,
            }).start();
          } else {
          	//Non c'è nessun immagine da visualizzare, viene effettuata l'animazione che riporta l'immagine alla posizione iniziale
            Animated.spring(this.state.panX, {
              toValue:0,
              velocity: gestureState.vx,
              tension:1,
              friction: 4
            }).start();
          }
        }
      }.bind(this));
    } else {
      //L'utente non ha completato lo swipe, l'immagine deve tornare alla posizione iniziale in modo animato
      Animated.spring(this.state.panX, {
        toValue:0,
        velocity: gestureState.vx,
       }).start();
    }
  }
});
\end{lstlisting}

Sfruttando le stesse API si è inoltre provato da implementare il pinch-to-zoom per permettere all'utente di ingrandire l'immagine.

Tuttavia il pinch-to-zoom è una gesture complessa che combina più gesture:
\begin{itemize}
\item pinch, per regolare lo zoom;
\item pan, per spostare l'immagine una volta zoomata;
\item swipe, per cambiare l'immagine.
\end{itemize}

\`E stata effettuata un'implementazione parziale di questa gesture che è risultata poco performante e pertanto si è deciso, in accordo con il tutor aziendale, di non proseguire lo sviluppo di tale funzionalità.

\section{Gestione degli errori}

Nell'ultima fase è stata implementata la gestione degli errori di connessione.

Le versioni precedenti dell'applicazione, infatti, non gestivano gli errori di comunicazione con il server, e nel caso se ne verificasse uno, i vari indicatori di attività rimanevano attivi fino al riavvio dell'applicazione.

\`E stato quindi implementato \texttt{ErrorsStore} in modo che fosse possibile memorizzare i vari messaggi d'errore all'interno dell'applicazione, inoltre, sono stati modificati tutti i metodi che creano delle azioni che necessitano di una connessione con il server, in modo che, se la promessa ritornata dai metodi di \texttt{WardaFetcher} fallisce, anziché eseguire il \textit{dispatch} normale dell'azione, richiedano ad \texttt{ErrorsActions} la creazione di un errore di rete, mediante il metodo \texttt{networkError()}.

La visualizzazione dei messaggi d'errore è stata affidata al componente di navigazione principale dell'applicazione, in modo che quando si verifichi un errore venga renderizzato il messaggio al posto della \textit{TabBar}.

\`E stato inoltre modificato il componente \texttt{NetworkImage} in modo che nel caso si verifichi un errore durante il caricamento dell'immagine venga visualizzato un messaggio d'errore al posto di un'immagine grigia.