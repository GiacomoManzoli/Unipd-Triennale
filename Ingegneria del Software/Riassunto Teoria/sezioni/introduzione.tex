\section{L'ingegneria del software}
L'ingegneria del software ha come scopo quello di soddisfare gli obbiettivi dati entro limiti accettabili di tempo e di sforzo, però non  è facile applicare principi ingegneristici al software.\\
Essa ha strette relazioni con svariate discipline sia informatiche che non (linguaggi di programmazione, architetture di sistemi operativi e di basi di dati, scienze gestionali..)\\
Un principio cardine dell'ingegneria del software fu illustrato da P. Brooks nel 1987 (\textit{``No Silver Bullet. Essence and Accidents of Software engineering''}) distinguendo tra:
\begin{itemize}
\item \textbf{Problematiche essenziali}: da usare come atteggiamento fondamentale nell'approccio al software, esse sono invasanti nel tempo della progettazione software. Esempi: specifica, realizzazione, verifica di prodotti software.
\item \textbf{Problematiche accidentali}: strumenti e tecniche per la rappresentazione e la verifica di accuratezza di rappresentazione delle problematiche essenziali. Esse mutano nel tempo e non sono sostanziali nello sviluppo software. Esempi: linguaggio di programmazione varia nel tempo.
\end{itemize}

Le problematiche accidentali possono essere rese sempre più agevoli dall'evoluzione tecnica e tecnologica ma il problema di analisi, di rigore, di astrazione fondamentali per poter risolvere le problematiche essenziali non potranno mai venire annullati.
\subsection{Definizioni di ingegneria del Software}
\textbf{Definizione IEEE}: L'approccio sistematico allo sviluppo, all'operatività, alla manutenzione e al ritiro del software --> il software è un prodotto con un proprio ciclo di vita.\\
\textbf{Definizione di Fairley} (1985): La disciplina tecnologica e gestionale per la produzione sistematica e la manutenzione di prodotti software sviluppati e modificati con tempi e costi preventivati --> Controllo della qualità.
\`{E} una complicata disciplina che racchiude varie tematiche tecnologiche connesse a tematiche economiche e manageriali. Essa affronta 3 tematiche principali:

\begin{itemize}
\item \textbf{Realizzazione di sistemi software}: attraverso strategie di analisi e progettazione;
\item \textbf{Processo software}: attraverso un'organizzazione e gestione dei progetti, l'individuazione di un ciclo di vita del software e l'applicazione di modelli astratti del processo di sviluppo;
\item \textbf{Qualità del software}: vengono previsti metodi di verifica e controllo della qualità secondo
metriche fissate che ne indicano la qualità.
\end{itemize}

\subsection{Software Engineer}
Il software engineer non è un programmatore ma una figura professionale che realizza parte di un sistema complesso che potrà essere usato, completato e modificato da altri. Egli deve guardare e comprendere in generale il quadro nel quale il suo sistema si colloca (che include il software) e saper operare compromessi tra visioni e spinte opposte (come costi, qualità, risorse..).

\subsection{Tipologie del software}
Le tipologie di prodotti software da produrre possono essere:
\begin{itemize}
\item Software su commessa (per specifici scopi);
\item Pacchetti software;
\item Componenti software (moduli integrati in altri software più complessi);
\item Servizi su sistemi e dati.
\end{itemize}
Molte volte però i progetti software subiscono forti ritardi o addirittura fallimenti dovuti o alle sbagliate analisi dei requisiti, cambi di tecnologie, esaurimento dei fondi, obsolescenza prematura.

\subsection{Manutenzione}
Tipi di manutenzione:
\begin{itemize}
\item \textbf{Correttiva}: per correggere difetti eventualmente rilevati;
\item \textbf{Adattativa}: per adattare il sistema a requisiti modificati (anche in corso d'opera);
\item \textbf{Evolutiva}: per aggiungere funzionalità al sistema (per migliorarlo);
\end{itemize}