\section{Qualità del software}
\begin{center}\textit{Dallo standard ISO 9000}\end{center}

\textbf{Qualità}: La capacità di un insieme di caratteristiche di un prodotto, sistema o processo, di soddisfare le esigenze dei clienti e degli altri portatori di interesse.\\
La percezione della qualità in campo software è arrivata molto tardi nonostante essa contribuisca valore alle attività. La qualità non è un concetto molto chiaro poiché può essere vista sotto aspetti differenti come il corretto soddisfacimento dei requisiti, l'idoneità all'uso del prodotto, la soddisfazione del cliente...\\
\textbf{Gestione della qualità}: L'insieme delle attività coordinate per dirigere e controllare un'organizzazione rispetto alla qualità.
\\
La qualità può essere gestita secondo diverse componenti quali la politica e gli obbiettivi di qualità, la pianificazione della qualità, il suo controllo ed accertamento, il suo miglioramento.\\
Se l'attività di gestione della qualità è ben pensata costituisce una politica che prevede una pianificazione a priori e un monitoraggio continuo, secondo il paradigma PDCA (processo che tende a migliorare con auto-verifica).\\

\textbf{Pianificazione della qualità}: Le attività della gestione della qualità mirate a definire gli obbiettivi della qualità ed i processi e le risorse necessarie per conseguirli.\\
La pianificazione della qualità è una premessa al controllo di essa secondo politiche corrette, scelte strategiche, strumenti di controllo e di gestione della qualità.\\

\textbf{Controllo della qualità}: Le attività della gestione della qualità messe in atto affinché il prodotto soddisfi i requisiti.
Il controllo può prevedere diverse modalità per la sua attuazione come il collaudo e la verifica oppure l'analisi e la conoscenza del dominio. L'enfasi in questa attività è legata alla gestione (vedo se quanto stiamo facendo è uguale a quanto pianificato). Fare controllo di qualità solo sul prodotto finito può essere molto rischioso quindi bisogna definire una tecnologia che dipani il controllo in corso d'opera, in momenti strategici, in modo da avere un controllo globale e continuo, per fare in modo di risolvere eventuali problemi.\\

\textbf{Accertamento della qualità}: Le attività della gestione della qualità messe in atto per accertare
che i requisiti siano soddisfatti.

L'accertamento è mirato al soddisfacimento degli obbiettivi. La percezione del livello di soddisfacimento può essere:
\begin{itemize}
\item \textit{Interna} --> da parte dell'azienda;
\item \textit{Esterna} --> da parte del cliente.
\end{itemize}
Si può parlare di qualità riferita a:
\begin{itemize}
\item prodotto (visione verticale): bene o servizio;
\item sistema: insieme degli elementi in cui il prodotto si colloca;
\item processi (visione orizzontale): attività correlate finalizzate alla realizzazione degli obbiettivi;
\item organizzazione: sulla struttura e l'amministrazione, a maggior ritorno economico.
\end{itemize}
La qualità di prodotto è legata alla qualità dei processi e non è mai rinchiusa all'interno del perimetro del prodotto (coinvolge anche attività esterne ad esso).
Da tenere presente che se un processo ha qualità allora anche i prodotti hanno qualità. (non viceversa).
Requisiti di qualità: Un esigenza od un'aspettativa dichiarata, comunemente intesa come implicita oppure obbligatoria (ISO 9000)

\subsection{Per avere buona qualità}
Definire bene cosa deve essere realizzato, come si controllerà\\
Controllare
\begin{itemize}
\item Per conoscere ed intervenire
\item Per dare/avere confidenza
\item Per migliorare i risultati
\end{itemize}

La qualità deve essere certificata con Norme per i prodotti, per tutelare il cliente rispetto all'uso od al valore di prodotti:
\begin{itemize}
\item FCC (Federal Communications Commission);
\item CE (Consumer Electronics);
\item OEM (Original Equipment Manufacturer);
\item DOC;
\item Carte dei servizi.
\end{itemize}

Norme per i processi:
\begin{itemize}
\item Requisiti di una funzione aziendale;
\item Ad es.: ISO 9001 per il sistema di qualità aziendale.
\end{itemize}
La norma esprime requisiti comuni.

Quali strumenti per garantire qualità?
\begin{itemize}
\item Seguire la ricetta (limitare la libertà creativa) definendo bene cosa fare (P-D) e controllare (C-A);
\item Analisi e definizione dei requisiti attraverso modelli per la qualità del software, strumenti per la definizione dei sistemi, metriche per definire livelli qualitativi;
\item Controllo continuo del progetto: rispetto dei vincoli contrattuali, controllo e verifica delle attività e dei risultati.
\end{itemize}

\subsection{Modelli della qualità software}
Valutazione dei prodotti: Visione dell'utente (problemi d'uso), visione dello sviluppatore (problemi tecnici), visione della direzione (problemi di costi)
Serve un solo modello per committenti e fornitori per uniformare la percezione della qualità e per uniformare la valutazione della qualità
\subsection{Modelli della qualità}
Strategia tipica: definizione di caratteristiche e loro organizzazione in una struttura logica
\textbf{Modello ISO/IEC 9126:2001} Software engineering - Product quality - Part1: Quality model\\
7 caratteristiche principali – 31 sottocaratteristiche
Strumento di definizione e valutazione che prevede organizzazione gerarchica delle caratteristiche e definizione di metriche
Visioni della qualità: interna, esterna, in uso -7 caratteristiche principali
visione del cliente: funzionalità, affidabilità, usabilità, efficienza, qualità in uso visione del fornitore: manutenibilità, portabilità
Il processo di valutazione:

%immagine  validazione

\section{Qualità del processo software}
Lo scopo è quello di
\begin{itemize}
\item  Definire il processo per controllarlo (e farlo controllare) meglio, per raccontarlo in maniera più convincente;
\item Controllare il processo per migliorarlo;
\item Efficacia: prodotti rispondenti ai requisiti.
\end{itemize}

Efficienza: minori costi a pari qualità di prodotto erogata --> Esperienza: apprendere dall'esperienza (anche degli altri)
\subsection{Le norme ISO 9000}
Certificazione ISO 9001 (2a metà anni '90) per valutare, per controllare, non per scegliere
La famiglia delle norme:
\begin{itemize}
\item 9000 Fondamenti e glossario; 
\item 9001 Sistema di Gestione della Qualità (SGQ) – requisiti;
\item 9004 Guida al miglioramento dei risultati.
\end{itemize}

\subsubsection{Documentazione del Sistema Gestione di Qualità (SGQ)}
\textbf{Manuale della qualità}: Il documento che definisce il sistema di gestione della qualità di un'organizzazione (ISO 9000)
\textit{Caratteristiche richieste}: completo rispetto ai requisiti, deve collegarsi al resto della documentazione del SGQ e delle procedure aziendali, deve relazionare gli obiettivi di qualità alle strategie per ottenerli ( esprimere la politica aziendale rispetto alla qualità).\\
\textbf{Piano della qualità}: Il documento che definisce gli elementi del SGQ e le risorse che devono essere applicati in uno specifico caso (prodotto, processo, progetto) (ISO 9000). \`E una concretizzazione specifica del Manuale della Qualità, ha spesso valenza contrattuale.
\textit{In pratica}: accerta la disponibilità di analisi dei requisiti, architettura e soluzioni tecniche, pianificazione delle verifiche e delle prove, risultati delle verifiche e delle prove. Fornisce modelli dei documenti, accertare la tracciabilità di soluzioni a requisiti e pianifica le attività
\textbf{Valutazione di un processo: modello CMM (CapabilityMaturityModel)}
Limiti del modello: una stessa realtà aziendale può adottare pratiche poste a livelli diversi e se non applica tutte le pratiche di un dato livello non può avanzare al livello superiore; modello discreto e non continuo incapace di differenziare tra L- ed L+; troppo focalizzato sulle pratiche (cosa si fa e come); insufficiente attenzione agli obiettivi (perché lo si fa)
Obiettivi di una valutazione
\begin{itemize}
\item I portatori d'interesse: i destinatari dei risultati, i responsabili dei processi valutati, i responsabili delle attività di valutazione;
\item Valutazione o miglioramento: risultato esterno o interno, valutazione formale o no (self-assessment);
\item Definizione della portata: dei processi inclusi nella valutazione e degli indicatori di valutazione.
\end{itemize}

