
\newcommand{\ao}{\mbox{Andrea} \mbox{Ongaro}\xspace}
\newcommand{\dm}{\mbox{Daniele} \mbox{Marin}\xspace}
\newcommand{\fv}{\mbox{Fabio} \mbox{Vedovato}\xspace}
\newcommand{\gma}{\mbox{Giacomo} \mbox{Manzoli}\xspace}
\newcommand{\gmi}{\mbox{Gianmarco} \mbox{Midena}\xspace}
\newcommand{\mb}{\mbox{Massimiliano} \mbox{Baruffato}\xspace}
\newcommand{\sm}{\mbox{Stefano} \mbox{Munari}\xspace}

%ruoli singolare per tabella
\newcommand{\rRPt}{Responsabile\xspace}
\newcommand{\rAPt}{Amministratore\xspace}
\newcommand{\rAt}{Analista\xspace}
\newcommand{\rPt}{Progettista\xspace}
\newcommand{\rVt}{Verificatore\xspace}
\newcommand{\rpt}{Programmatore\xspace}

%ruoli singolare
\newcommand{\rRP}{\emph{Responsabile di Progetto}\xspace}
\newcommand{\rAP}{\emph{\rAPt}\xspace}
\newcommand{\rA}{\emph{\rAt}\xspace}
\newcommand{\rP}{\emph{\rPt}\xspace}
\newcommand{\rV}{\emph{\rVt}\xspace}
\newcommand{\rp}{\emph{\rpt}\xspace}

%ruoli plurale
\newcommand{\rRPs}{\emph{Responsabili di Progetto}\xspace}
\newcommand{\rAPs}{\emph{Amministratori}\xspace}
\newcommand{\rAs}{\emph{Analisti}\xspace}
\newcommand{\rPs}{\emph{Progettisti}\xspace}
\newcommand{\rVs}{\emph{Verificatori}\xspace}
\newcommand{\rps}{\emph{Programmatori}\xspace}

%revisioni
\newcommand{\RR}{\textbf{Revisione dei Requisiti}\xspace}
\newcommand{\RP}{\textbf{Revisione di Progettazione}\xspace}
\newcommand{\RQ}{\textbf{Revisione di Qualifica}\xspace}
\newcommand{\RA}{\textbf{Revisione di Accettazione}\xspace}

%documenti
\newcommand{\LP}{\emph{Lettera di Presentazione}\xspace}
\newcommand{\AR}{\emph{Analisi dei Requisiti}\xspace}
\newcommand{\G}{\emph{Glossario}\xspace}
\newcommand{\NP}{\emph{Norme di Progetto}\xspace}
\newcommand{\PP}{\emph{Piano di Progetto}\xspace}
\newcommand{\PQ}{\emph{Piano di Qualifica}\xspace}
\newcommand{\SF}{\emph{Studio di Fattibilità}\xspace}
\newcommand{\ST}{\emph{Specifica Tecnica}\xspace}
\newcommand{\MU}{\emph{Manuale Utente}\xspace}
\newcommand{\DP}{\emph{Definizione di Prodotto}\xspace}

%fasi
\newcommand{\fA}{\textbf{\fAt}\xspace}
\newcommand{\fAD}{\textbf{\fADt}\xspace}
\newcommand{\fPA}{\textbf{\fPAt}\xspace}
\newcommand{\fPD}{\textbf{\fPDt}\xspace}
\newcommand{\fC}{\textbf{\fCt}\xspace}
\newcommand{\fVV}{\textbf{\fVVt}\xspace}

\newcommand{\fAt}{\mbox{Ammissione} al \mbox{progetto}\xspace}
\newcommand{\fADt}{\mbox{Consolidamento} dei \mbox{requisiti}\xspace}
\newcommand{\fPAt}{\mbox{Progettazione} \mbox{dell'architettura}\xspace}
\newcommand{\fPDt}{\mbox{Consolidamento} dell'\mbox{architettura}\xspace}
\newcommand{\fCt}{\mbox{Realizzazione} del \mbox{prodotto}\xspace}
\newcommand{\fVVt}{\mbox{Collaudo} \mbox{finale}\xspace}

\newcommand{\scopoProdotto}{Lo scopo del prodotto è di permettere la creazione e l'esecuzione di presentazioni a partire da \gloxy{mappe mentali}. 
L'utente sarà guidato nella creazione di una \gloxy{mappa mentale} e di uno o più \gloxy{percorsi di presentazione}, utilizzando i nodi di tale mappa. 
L'utente potrà eseguire una presentazione seguendo un \emph{percorso} creato oppure visitando qualsiasi nodo della \emph{mappa} costruita; rompendo così la sequenzialità nella presentazione.
Il prodotto sarà utilizzabile attraverso un \gloxy{browser}.} 

\newcommand{\descrizioneGlossario}{Al fine di evitare ogni ambiguità di linguaggio e massimizzare la comprensione dei
documenti, i termini tecnici, di dominio, gli acronimi e le parole che necessitano di
essere chiarite, sono riportate nel documento \glossario.
Ogni occorrenza dei vocaboli presenti nel \G è marcata da una ``G'' maiuscola in
pedice ed è scritta in corsivo (es: \gloxy{Esempio}).}

\newcommand{\analisiDeiRequisiti}{\AR\emph{v\input{../analisiDeiRequisiti/versione.tex}}\xspace}
\newcommand{\glossario}{\G\emph{v\input{../glossario/versione.tex}}\xspace}
\newcommand{\normeDiProgetto}{\NP\emph{v\input{../normeDiProgetto/versione.tex}}\xspace}
\newcommand{\pianoDiProgetto}{\PP\emph{v\input{../pianoDiProgetto/versione.tex}}\xspace}
\newcommand{\pianoDiQualifica}{\PQ\emph{v\input{../pianoDiQualifica/versione.tex}}\xspace}
\newcommand{\studioDiFattibilita}{\SF\emph{v\input{../studioDiFattibilita/versione.tex}}\xspace}
%\newcommand{\specificaTecnica}{\ST\emph{v\input{../specificaTecnica/versione.tex}}\xspace}
%\newcommand{\manualeUtente}{\MU\emph{v\input{../manualeUtente/versione.tex}}\xspace}
%\newcommand{\definizioneDiProdotto}{\DP\emph{v\input{../definizioneDiProdotto/versione.tex}}\xspace}

\newcommand{\vEsternoDicembre}{\i1} %per retrocompatibilità

\newcommand{\iI}{\emph{I1 v\input{../20141219I1/versione.tex}}\xspace} %per motivi di leggibilità la I non è in corsivo
\newcommand{\iII}{\emph{I2 v\input{../20150313I2/versione.tex}}\xspace}
\newcommand{\eI}{\emph{E1 v\input{../20150302E1/versione.tex}}\xspace}
\newcommand{\eII}{\emph{E2 v\input{../20150302E2/versione.tex}}\xspace}


\newcommand{\proponente}{\mbox{Zucchetti} S.p.A.\xspace}
\newcommand{\referenteProponente}{\mbox{Gregorio} \mbox{Piccoli}\xspace}
\newcommand{\committente}{Prof. \mbox{Vardanega} \mbox{Tullio}\xspace}
\newcommand{\committenteAlt}{Prof. \mbox{Cardin} \mbox{Riccardo}\xspace}
\newcommand{\gruppo}{Pragma\xspace}
\newcommand{\progetto}{Premi\xspace}
\newcommand{\groupmail}{\url{pragma.swe@gmail.com}\xspace}
\newcommand{\pragmadb}{\emph{PragmaDB}\xspace}
\newcommand{\pragmaDocs}{\emph{pragmaDocs}\xspace}
\newcommand*{\customRef}[2]{\hyperref[{#1}]{#2 \ref*{#1}}}
\newcommand*{\hRef}[2]{\hyperref[{#1}]{#2}}

\newcommand{\nogloxy}[1]{#1} % comando da usare per evitare di metttere il mark del glossario
\newcommand{\gloxy}[1]{\emph{#1}$_G$}


\newcommand{\diaryEntry}[5]{#1 & \emph{#4} & #3 & #5 & #2\\ \hline}
\newcommand{\capitalizeFirstLetter}[1]{\StrLeft{#1}{1}[\temp]\uppercase\expandafter{\temp}\StrLen{#1}[\temp]\StrMid{#1}{2}{\temp}}
\newcommand{\conversationEntry}[2]{\textbf{\capitalizeFirstLetter{#1}}\\\\\emph{\indent \capitalizeFirstLetter{#2}}}
\newcommand{\chrule}{\begin{center}\line(1,0){250}\end{center}}

\newcommand{\si}{\rule{0pt}{4.8ex}\includegraphics[width=0.5cm]{../template/icone/yes.pdf}\xspace}
\newcommand{\no}{\rule{0pt}{5.3ex}\includegraphics[width=0.5cm]{../template/icone/no.pdf}\xspace}

\newcommand{\version}{\input{versione.tex}}