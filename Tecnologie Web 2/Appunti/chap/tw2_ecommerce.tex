\section{E-Commerce}
In un sito e-commerce la cosa più importante è la vicinanza tra il prezzo e il prodotto, infatti agli utenti piace vedere sia il prodotto che il prezzo contemporaneamente.
Quando per vedere il prezzo è necessario cliccare sull'immagine del prodotto o su un link, si crea una situazione di gambling (\textbf{Gambling click}) nella quale l'utente clicca a caso sperando di trovare il prezzo il che fa diminuire il gradimento da parte dell'utente del 40\%.

\subsection{Farsi pubblicità}
Dato che l'utente non dedica molto attenzione alle pubblicità, si ha poco tempo per impressionarlo.
Comunemente per invogliare l'utente viene visualizzato un prezzo volutamente più basso detto \textbf{Fishing price} oppure un prezzo che esclude le tasse doganali e/o le spese di spedizione (\textbf{Net price}).
Il problema di queste tecniche è che sono pensate per la pubblicità classica, dove l'utente vede la pubblcità e associa al negozio il fatto che ha un prezzo vantaggiso dimenticandosi del prezzo preciso, nel web invece il tempo che passa tra quando l'utente vede la pubblicità e visita il negozio è molto più breve e quindi riesce a riconoscere l'inganno del prezzo.\\
Il 90\% degli utenti che se ne accorge abbandona il sito.\\
Una buona idea è quella di visualizzare una stima del prezzo e poi nel carrello mostrare il prezzo corretto compreso di spese di spedizione, inoltre per i siti che offrono qualcosa di gratuito è bene specificarlo, la parola gratis piace agli utenti.

\subsection{La descrizione del prodotto}
Uno degli errori più grandi che possono fare i proprietari dei siti e-commerce è quello di assumere che il prodotto sia già conosciuto da parte dell'utente.
Se l'utente non trova una descrizione del prodotto che lo soddisfa probabilmente andrà a cercarla su un altro sito, il 95\% degli utenti preferisce acquistare nel sito dove trova la descrizione dettagliata ed è persino disposto a pagare fino al 20\% in più.

\subsubsection{Le foto del prodotto}
Se l'utente è interessato a vederle è bene che siano presenti dato che, oltre a rendere il sito più professionale, i timer dell'utente si bloccano mentre le guarda perché la sua attenzione e posta sui dettagli del prodotto e non sulla pagina web.
Tutto questo funziona se la vista dei dettagli è opzionale e avviene nella stessa pagina.