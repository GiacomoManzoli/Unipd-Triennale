\section{Eyetrack}
Studio realizzato mediante un eyetracker che registra dove guarda l'utente. Questo studio riguarda come gli utenti agiscono durante la fase di scanning, per mostrare i dati sono state usate \textbf{mappe termiche} con colori più caldi per le zone osservate per più tempo da parte degli utenti.
Tra i vari risultati è emerso che l'utente tende a concentrare di più lo sguardo nella parte in alto a sinsitra del sito, per poi spostarsi da sinistra verso destra e una volta finita la parte superiore verso il basso.
Un'altro sorprendete risultato è stato che gli utenti tendono a concentrarsi più sul testo che sulle immagini, questo perché nel web le immagini sono tendenzialmente o di abbellimento o pubblicità.

\subsection{Testo e paragrafi}
Si è osservato che gli utenti tendono a scansionare meglio il testo quando è su una singola colonna ed è diviso in piccoli paragrafi, inoltre il testo scritto in piccolo, purché sia leggibile, tende ad incentivare la lettura mentre il testo scritto più in grande viene solo scansionato. \\
Per quanto riguarda i titoli, più sono corti, più attirano l'attenzione, si è notato anche che gli utenti leggono la parte più a sinistra del \textit{blurb} (sommario) solo se è scritto come il titolo e non ci sono separazioni.	\\
Le keyword all'interno dei paragrafi sono utili per creare degli hot spot che catturano l'attenzione dell'utente, però non basta che siano semplicemente in bold, devono anche essere sottolineate e di dimensione maggiore.

\subsection{Le immagini}
Anche se l'utente preferisce il testo alle immagini, anche queste vengono visualizzate, se si vuole che un'immagine venga visualizzata è bene che sia almeno 210x210 pixel.
Si è notato che gli utenti tendono comunque a cliccare sopra le immagini anche se ciò non provoca nessun effetto.
Ci sono dei casi particolari in cui le immagini rendono meglio del testo, cioè quando c'è da spiegare all'utente un concetto nuovo, meglio ancora se accompagnato da una spiegazione audio, si è però notato anche che se venogno usate immagini, audio e testo l'utente tende ad ignorare il testo.

\subsection{Layout e Navigazione}
Tra i vari layout testati si è notato che la barra di navigazione è stata notata meglio quando è nella parte alta della pagina e che layout più separati facilitano la scansione della pagina mentre layout compatti sono più efficaci quando l'utente deve leggere il contenuto perché diminuiscono lo scrolling.
Sempre riguardo lo scrolling si è notato che gli utenti scrollano parte della pagina ma la scansionano velocemente cercando qualcosa che attiri la loro attenzione.

\subsection{Pubblicità}
Molto spesso gli utenti tendono ad ignorare la pubblicità e qui pochi che ci fanno caso tendono a dedicarci circa 1 secondo.
Si è notato che i posti migliori dove metterele sono in alto a sinistra oppure in mezzo a del contenuto interessante, è importante però che tra la pubblicità e il contenuto non ci siano margini o separatori. \\
Bisogna stare attenti anche a non abusare della pubblciità altrimenti l'utente si stanca prima e la sua voglia di ritornare nel sito diminuisce dell'80\%.\\
Per ottenere il massimo da una pubblicità conviene che sia scritta sotto forma testuale e che sia grande, inoltre aiuta molto il \textbf{Behavioral Advertising} cioè fornire pubblicità basate sulla cronologia dell'utente ,questo tipo di pubblcità è più apprezzato dagli utenti perché riguarda cose che gli interessano (fino a 100 volte più efficace e aumenta la voglia di ritorno).\\
Altre caratteristiche delle pubblicità\textbf{odiate} dalla maggior parte degli utenti sono:
\begin{itemize}
\item suoni che parto in automatico, 79\%
\item lampeggiamenti vari, 87\%
\item occupano gran parte dello schermo, 90\%
\item si spostano sullo schermo, 92\%
\item non chiare, 92\%
\item coprono il contenuto, 93\%
\item non si possono chiudere, 93\%
\item cercano di farsi cliccare, 94\%
\item si caricano lentamente, 94\% \begin{quote}
Posso sopportare i video che si devono caricare, posso sopportare le pubblicità, ma quando le pubblicità si devono caricare...
\end{quote}
\item pop up, 95\%
\end{itemize}


\section{Legge di Fitts}
$$ T = a + b\cdot\log_{2}(1 + \frac{D}{W}) $$
Dove:
\begin{itemize}
\item \textbf{T} è il tempo che l'utente impiega a spostare il mouse da un punto ad un altro della pagina
\item \textbf{a} rappresenta il tempo di start/stop dell'utente
\item \textbf{b} è la lentezza del movimento, covelocità
\item \textbf{D} è la distanza da percorrere
\item \textbf{W} è la dimesione della destinazione
\end{itemize}
Questa regola è importantissima perché, oltre ad essere corretta nel 98\% dei casi, ci aiuta molto a capire come si comporta l'utente permettendoci di fare alcune considerazioni riguardo l'usabilità delle varie interfacce grafiche.
Alcune conseguenze sono:
\begin{itemize}
\item Il drag'n'drop è sconsigliabile perché viene eseguito più lentamente e stanca prima l'utente, anche per la tensione muscolare necessaria per fare il movimento.
\item I classici menù a tendina sono visti male dall'utente perché certe volte li chiude per sbaglio seguento il cammino minimo. \\
Si sono creati quindi i menù \textbf{bilanciati} nei quali i sottomenù venogno mostrati centrati. Questo perché il web è uno spazio bidimensionale e di conseguenza conta anche la traiettoria.
\item \textbf{Pie Menù} sono dei menù a torta che vengono visualizzati nel punto in cui si trova il mouse diminuendo notevolemente la distanza che l'utente deve compiere per selezionare una voce. Anche se sono più pratici questi menù hanno dei problemi dovuti al fatto che possono visualizzare poche e semplici informazioni, altrimenti gli spicchi diventerebbero troppo piccoli.
\end{itemize}
Una grande spinta verso nuove interfacce ottimizzate secondo la legge di Fitts viene dal mondo dei videogiochi, mondo nel quale l'usabilità è un fattore chiave e nel quale sono disponibili investimenti più sostanziosi.

\subsection{Target Size Rule}
La taglia dei bottoni dovrebbe essere proporzionale alla loro frequenza di utilizzo, come nella suite Office.\\
Un'applicazione ancora più efficacie è l'utilizzo dei bottoni a bordo schermo o negli angoli, in questo l'utente può andare a "sbattere" contro il bordo senza dover fare lo sforzo di fermarsi. Questo fatto si riscontra anche nella formula di Fitts in quanto si può vedere il bordo come un bottone di dimensione infinita, ne segue anche che un bottone a bordo è di più facile utilizzo rispetto ad uno a pochi pixel dal mouse, questo perché il tempo di start/stop influenza maggiormente il tempo totale.

\section{Ricerca}
Per i siti di grosse dimensioni è fondamentale avere un ottimo tool di ricerca, può fare comodo anche su siti più piccoli perché può essere che l'utente sia arrivato al sito trammite \textbf{deep linking} e che quindi non sappia dove trovare l'informazione.\\
Per avere un tool di ricerca si può usare un motore di ricerca già esistenste in modo che esegua la ricerca solamente in locale, questa soluzione però è limitata dal fatto i motori non indicizzano tutto il sito e che se l'utente è arrivato al sito mediante un motore di ricerca, l'utilizzo dello stesso motore di ricerca per cercare in locale produrrà gli stessi risultati.\\
Inoltre può capitare che l'utente venga reindirizzato verso la SERP del motore, portandolo fuori dal sito.\\
Conviene allora creare un proprio tool locale, che rispetti le aspettative dell'utente, cioè che sia simile a quello dei motori di ricerca, seguendo la filosofia \textbf{less is more} per evitare di conforndere l'utente.

\subsection{Come strutturare la ricerca}
Anzitutto bisogna decidere se offrire una ricerca \textbf{classica} o con \textbf{vincoli}, quella con vincoli piace di più agli utenti perché è più efficace, in ogni caso è buona norma offrire entrambi i tipi di ricerca.
La ricerca con vincoli può essere \textbf{statica}, quando l'utente prima inserisce i dati e poi preme cerca, oppure \textbf{dinamica}, che aggiorna i risultati man mano che l'utente inserisce i dati.\\
Entrambe hanno i propri pro e contro, la ricerca statica crea confusione negli utenti perché associano il pulsante cerca alla ricerca classica mentre la ricerca dinamica richiede più tempo per visualizzare i risultati e un maggior sforzo da parte del server.\\
Conviene quindi usare quella dinamica quando ci sono pochi vincoli in caso contrario è meglio quella statica oppure si può usare una versione ibrida nella quale la ricerca parte in automatico quando sono stati impostati tutti i vincoli mentre è statica se alcuni vincoli non sono settati.

\subsection{Box di ricerca}
La misura ideale del box di ricerca è di 30 caratteri, in questo modo viene accontentato circa il 90\% degli utenti.
Dimensionare correttamente il box è importante perché se è troppo piccolo può verificarsi lo scroll orizzontale e inoltre tende ad influenzare l'utente il quale si sente invitato ad inserire query corte che possono influenzare la qualità dei risultati.

\subsubsection{I risultati}
Per visualizzare i risultati si può usare un griglia o una lista, la griglia sembrerebbe ottimale dato che è più compatta e quindi nello stesso spazio vengono visualizzati più risultati, però l'utente per analizzare i risultati disposti in una griglia tende a seguire un cammino casuale che porta ad una perdita di tempo e ad un maggior sforzo computazionale.
Nel lato pratico vengono quindi usate le liste.

\subsection{404: Title not found}
La funzionalità di ricerca gioca un ruolo importante anche quando l'utente visita un link rotto, infatti in questi casi la soluzione migliore è mostrare all'utente sia il messaggio che spiega l'errore sia il tool di ricerca in modo da offrirgli un piano B.
Altre idee come lasciare l'errore di default oppure reindirizzare alla home creano solo confusione/frustrazione.

\subsection{La ricerca 2.0}
Fornire tool di ricerca elaborati con interfacce stile umanoide fa diminuire del 43\% la soddisfazione dell'utente, questo perché un'interfaccia avanzata ne aumenta notevolmente le aspettative.
Se l'umanoide è più un cartone animato, quindi si vede che è virtuale, l'aspettativa cala e di conseguenza il gradimento sale.\\
Ikea come alternativa alla ricerca classica, fornisce anche una ricerca 2.0 con una interfaccia 2D  fatta in stile cartone animate che abbassa le aspettative dell'utente, aumentandone il gradimento fino al +70\%.



