\section{Usabilità}

\subsection{Homepage}
L'homepage è la pagina principale del sito, l'equivalente della vetrina di un negozio, è dunque importante che sia chiara e ben fatta.
Idealemente l'homepage dovrebbe rispondere ad una versione modificate delle 6W del giornalismo:
\begin{itemize}
\item \textbf{Where} In che sito sono arrivato? In che parte del sito sono?
\item \textbf{Who} Chi rappresenta il sito? Chi ha scritto il contenuto?
\item \textbf{Why} Perché l'utente dovrebbe restare nel sito?
\item \textbf{When} Quali sono le news? Quanto aggiornato è il contenuto?
\item \textbf{What} Di che cosa parla il sito?
\item \textbf{How} Come si può muovre l'utente nel sito?
\end{itemize}
Rispondere a queste domande in modo efficace e in poco tempo è una cosa cruciale dato che normalmente l'utente medio dedica circa 30 secondi alla homepage, tempo che diminuisce ancora se l'utente ha già visualizzato la pagina un'altra volta.
\\
Quando l'utente passa dalla home ad un'altra pagina non sono più necessari tutti gli assi informativi dal momento che si è creato una sorta di legame tra l'utente e il sito.
Grazie a questo legame l'utente è disposto a passare più tempo in una pagina interna (50 secondi in media).

\subsubsection{I timer}
L'utente medio è impaziente, non dedica tanto tempo ad un sito web. Per valutare se vale la pena coninuare con la naviazione del sito l'utente impiega circa 1 minuto e 50 secondi (\textbf{tempo preliminare}) dopodiché se non è soddisfatto abbandona il sito e con il 90\% di possibilità non tornerà mai più, indipendentemente dalla presenza o meno dei contenuti.
\\
Nel caso l'utente decida di restare, il suo \textbf{tempo complessivo} di permanenza non superà comunque i 4 minuti, allo scadere dei quali se non avrà trovato quello che cercava se ne andrà comunque insoddisfatto. Di conseguenza è cruciale che le informazioni siano facilemente reperibili, una buona norma è non superare mai i 3 click dalla homepage.

\subsubsection{Deep Linking}
Se un utente arriva al sito trammite un motore di ricerca è molto probabile che non passi dalla homepage, perciò è importante che anche le pagine interne del sito rispondano ad alcune delle 6W della homepage.
In particolare l'utente dovrebbe sempre riuscire a sapere chi siamo, come tornare alla home e in che punto è nel sito.
Una pratica comune è l'inserimento delle \textbf{Breadcrumbs}, serie di link che rispecchiano il percorso che l'utente avrebbe dovuto fare per arrivare nella pagina.
Ci sono vari tipi di breadcrumbs:
\begin{itemize}
\item \textbf{Location} indicano dove ci si trova nel sito a partire dalla homepage
\item \textbf{Attribute} indica il percorso logico che ha portato dalla homepage alla pagina corrente, non rispecchia necessariamente quello delle pagine
\item \textbf{Path} indica il cammino fisico che ha fatto l'utente all'interno del sito, è di tipo dinamico e viene visualizzato dalla seconda pagina in poi.
\end{itemize}

\subsection{Problemi di navigazione}
Si verificano quando l'asse informativo \textit{where} non è ben sviluppato, cioè quando l'utente non riesce a capire dove si trova rispetto alla home (\textbf{lost in navigation}).
Per risolvere questi problemi si possono usare le breadcrumbs che funzionano abbastanza bene anche se non mostrano il percoso che ha fatto l'utente all'interno del sito, per questo motivo ai link dovrebbero essere associato un colore diverso se sono già stati visitati dall'utente dimunuendone così lo sforzo computazionale. \\
Un'altra cosa da tenere in considerazione è il \textbf{backtracking}, cioè la continua visita di un sito usando frequentemente la funzinalità di ritorno del browser. Questa funzionalità piace tanto agli utenti al punto che preferiscono premere 7 volte il pulsante del browser per tornare alla home piuttosto che usare un link diretto, questo perché richiede meno sforzo computazionale.\\
Bisogna tenere conto di ciò anche quando si aprono i link su nuove finestre o schede diverse, pratica da evitare assolutamente dato che "rompe" il backtracking. L'utizzo di nuove finestre è anche sconsigliato perché se la nuova finestra finisce in background l'utente può non accorgersene e rimanerne frustatrato.

\subsection{Legge di Jakob}
\begin{quote}
Gli utenti passano la maggior parte del tempo su altri siti.
\end{quote}
Quindi è conveniente attenersi alle convenzioni per agevolare la visita degli utenti.

 

