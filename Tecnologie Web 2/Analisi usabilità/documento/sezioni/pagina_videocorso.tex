\subsection{Vendita: Corso Lightroom}\label{lightroom}
Lo scopo di questa pagina è convincere l'utente ad acquistare il corso di Lightroom 5 realizzato dal curatore del sito.\\
Questa pagina ha un grande problema problema: \textbf{il prezzo}.\\
Il prezzo del corso viene scritto in fondo alla pagina e per raggiungerlo il visitatore deve scrollare per la lunghezza di ben \textbf{9} pagine\footnote{Considerando uno schermo a risoluzione 1920x1080}. Questo fa si che l'utente legga tutta la descrizione del corso in modo distratto, perché la sua curiosità principale sarà:
\begin{center}
\textit{ma quanto mi costa?}
\end{center}

Un approccio decisamente migliore sarebbe quello di elencare una breve lista delle caratteristiche del corso con posto vicino il prezzo e solo successivamente descrivere in modo dettagliato i benefici che l'utente può ricavarne dall'acquisto.
Ciò è ottenibile semplicemente spostando la parte \textit{Ricapitoliamo} della pagina in cima, in modo che l'utente riesca a sapere subito il contenuto del corso e il prezzo.

L'ultima osservazione riguardo questa pagina è sull'immagine introduttiva, la quale risulta estremamente inefficace dal momento che la scelta dei colori la rende troppo simile ad una pubblicità, correndo il rischio che il visitatore la ignori, portando ad un inutile spreco di spazio.