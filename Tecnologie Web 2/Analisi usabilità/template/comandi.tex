
\newcommand{\ao}{\mbox{Andrea} \mbox{Ongaro}\xspace}
\newcommand{\dm}{\mbox{Daniele} \mbox{Marin}\xspace}
\newcommand{\fv}{\mbox{Fabio} \mbox{Vedovato}\xspace}
\newcommand{\gma}{\mbox{Giacomo} \mbox{Manzoli}\xspace}
\newcommand{\gmi}{\mbox{Gianmarco} \mbox{Midena}\xspace}
\newcommand{\mb}{\mbox{Massimiliano} \mbox{Baruffato}\xspace}
\newcommand{\sm}{\mbox{Stefano} \mbox{Munari}\xspace}

%ruoli singolare per tabella
\newcommand{\rRPt}{Responsabile\xspace}
\newcommand{\rAPt}{Amministratore\xspace}
\newcommand{\rAt}{Analista\xspace}
\newcommand{\rPt}{Progettista\xspace}
\newcommand{\rVt}{Verificatore\xspace}
\newcommand{\rpt}{Programmatore\xspace}

%ruoli singolare
\newcommand{\rRP}{\emph{Responsabile di Progetto}\xspace}
\newcommand{\rAP}{\emph{\rAPt}\xspace}
\newcommand{\rA}{\emph{\rAt}\xspace}
\newcommand{\rP}{\emph{\rPt}\xspace}
\newcommand{\rV}{\emph{\rVt}\xspace}
\newcommand{\rp}{\emph{\rpt}\xspace}

%ruoli plurale
\newcommand{\rRPs}{\emph{Responsabili di Progetto}\xspace}
\newcommand{\rAPs}{\emph{Amministratori}\xspace}
\newcommand{\rAs}{\emph{Analisti}\xspace}
\newcommand{\rPs}{\emph{Progettisti}\xspace}
\newcommand{\rVs}{\emph{Verificatori}\xspace}
\newcommand{\rps}{\emph{Programmatori}\xspace}

%revisioni
\newcommand{\RR}{\textbf{Revisione dei Requisiti}\xspace}
\newcommand{\RP}{\textbf{Revisione di Progettazione}\xspace}
\newcommand{\RQ}{\textbf{Revisione di Qualifica}\xspace}
\newcommand{\RA}{\textbf{Revisione di Accettazione}\xspace}

%documenti
\newcommand{\AR}{\emph{Analisi dei Requisiti}\xspace}
\newcommand{\G}{\emph{Glossario}\xspace}
\newcommand{\NP}{\emph{Norme di Progetto}\xspace}
\newcommand{\PP}{\emph{Piano di Progetto}\xspace}
\newcommand{\PQ}{\emph{Piano di Qualifica}\xspace}
\newcommand{\SF}{\emph{Studio di Fattibilità}\xspace}
\newcommand{\ST}{\emph{Specifica Tecnica}\xspace}
\newcommand{\MU}{\emph{Manuale Utente}\xspace}
\newcommand{\DP}{\emph{Definizione di Prodotto}\xspace}

%fasi
\newcommand{\fA}{\textbf{\fAt}\xspace}
\newcommand{\fAD}{\textbf{\fADt}\xspace}
\newcommand{\fPA}{\textbf{\fPAt}\xspace}
\newcommand{\fPD}{\textbf{\fPDt}\xspace}
\newcommand{\fC}{\textbf{\fCt}\xspace}
\newcommand{\fVV}{\textbf{\fVVt}\xspace}

\newcommand{\fAt}{\mbox{Analisi}\xspace}
\newcommand{\fADt}{\mbox{Analisi di Dettaglio}\xspace}
\newcommand{\fPAt}{\mbox{Progettazione Architetturale}\xspace}
\newcommand{\fPDt}{\mbox{Progettazione di Dettaglio}\xspace}
\newcommand{\fCt}{\mbox{Codifica}\xspace}
\newcommand{\fVVt}{\mbox{Verifica e Validazione}\xspace}


%scopo prodotto, glossario
\newcommand{\scopoProdotto}{Lo scopo del prodotto è quello di fornire all'utente la possibilità di creare una mappa mentale e di poterla usare per costruire dei percorsi di visualizzazione dei nodi della mappa. \\
Il prodotto dovrà funzionare sul browser del computer dell'utente, mentre le presentazioni create potranno essere visualizzate anche su dispositivi mobile quali smartphone e tablet.} 

%vecchia versione
%Lo scopo del progetto è la realizzazione di un’applicazione di presentazione di slide innovativa, che offra la possibilità di creare, visualizzare e stampare le presentazioni.
%L’applicazione dovrà funzionare ed essere accessibile mediante l’utilizzo del browser e funzionare anche su dispositivi mobile quali smartphone e tablet.


%\newcommand{\descrizioneGlossario}{Per evitare ogni genere di ambiguità relativa al linguaggio e ai termini utilizzati nei documenti formali, cercando di massimizzare la comprensione dei documenti nella loro interezza, si allega il \glossario , documento nel quale vengono descritti tutti i termini che necessitano chiarimento.\\ \\Per evidenziare i termini persenti nel glossario verrà posta a fianco ad essi una 'G' in pedice (esempio: esempio$_G$).}

\newcommand{\descrizioneGlossario}{Al fine di evitare ogni ambiguità di linguaggio e massimizzare la comprensione dei
documenti, i termini tecnici, di dominio, gli acronimi e le parole che necessitano di
essere chiarite, sono riportate nel documento \glossario.
Ogni occorrenza di vocaboli presenti nel \G è marcata da una ``G'' maiuscola in
pedice.}



%documenti con versione, chi approva deve aggiornare il numero di versione.
\newcommand{\analisiDeiRequisiti}{\AR\emph{v1.2.0}\xspace}
\newcommand{\glossario}{\G\emph{v1.2.0}\xspace}
\newcommand{\normeDiProgetto}{\NP\emph{v1.2.0}\xspace}
\newcommand{\pianoDiProgetto}{\PP\emph{v1.2.0}\xspace}
\newcommand{\pianoDiQualifica}{\PQ\emph{v1.2.0}\xspace}
\newcommand{\studioDiFattibilita}{\SF\emph{v1.2.0}\xspace}
\newcommand{\specificaTecnica}{\ST\emph{vX.Y.Z}\xspace}
\newcommand{\manualeUtente}{\MU\emph{vX.Y.Z}\xspace}
\newcommand{\definizioneDiProdotto}{\DP\emph{vX.Y.Z}\xspace}
\newcommand{\vEsternoDicembre}{\emph{Verbale esterno 2014-12-19 v1.2.0}\xspace}

\newcommand{\proponente}{\mbox{Zucchetti} S.p.A.\xspace}
\newcommand{\referenteProponente}{\mbox{Gregorio} \mbox{Piccoli}\xspace}
\newcommand{\committente}{Prof. \mbox{Vardanega} \mbox{Tullio}\xspace}
\newcommand{\committenteAlt}{Prof. \mbox{Cardin} \mbox{Riccardo}\xspace}
\newcommand{\gruppo}{Pragma\xspace}
\newcommand{\progetto}{Premi\xspace}
\newcommand{\groupmail}{pragma.swe@gmail.com\xspace}
\newcommand{\pragmadb}{\emph{PragmaDB}\xspace}

\newcommand{\nomark}[1]{#1} % comando da usare per evitare di metttere il mark del glossario


\newcommand{\g}[1]{\emph{#1}$_G$}
\newcommand{\diaryEntry}[5]{#1 & \emph{#4} & #3 & #5 & #2\\ \hline}
\newcommand{\capitalizeFirstLetter}[1]{\StrLeft{#1}{1}[\temp]\uppercase\expandafter{\temp}\StrLen{#1}[\temp]\StrMid{#1}{2}{\temp}}
\newcommand{\conversationEntry}[2]{\textbf{\capitalizeFirstLetter{#1}}\\\\\emph{\indent \capitalizeFirstLetter{#2}}}
\newcommand{\chrule}{\begin{center}\line(1,0){250}\end{center}}

\newcommand{\si}{\rule{0pt}{4.8ex}\includegraphics[width=0.5cm]{../template/icone/yes.pdf}\xspace}
\newcommand{\no}{\rule{0pt}{5.3ex}\includegraphics[width=0.5cm]{../template/icone/no.pdf}\xspace}